%!TEX program=xelatex
\documentclass{article}
%packages
\usepackage[UTF8]{ctex} %中文字符
\CTEXoptions[today=old]
\usepackage[left=2.50cm, right=2.50cm, top=2.50cm, bottom=2.50cm]{geometry} %调整页边距
\usepackage{xeCJK}
\usepackage{enumerate}%使用枚举
\usepackage{mathtools}
\usepackage{titlesec}
\renewcommand{\thesection}{\chinese{section}}%设置section
\usepackage{bm}%加粗数学公式
\usepackage{amssymb}%因为所以
\usepackage{mathrsfs} %花体字
\usepackage{amsmath}%不太花的花体字
% MATLAB代码插入包
\usepackage{listings}
\usepackage[framed,numbered,autolinebreaks,useliterate]{mcode}

\begin{document}
\title{《信号检测与估计》作业三 \footnote{习题4.1\quad4.2\quad4.3\quad4.4}}
\author{施念  1120161302}
\date{May 9, 2019}
\maketitle{}
%\begin{abstract}%摘要部分
%\end{abstract}
%\tableofcontents%生成目录
\section{第4章\ 估计的基本理论}
\begin{enumerate}[1.]
%%%%%%%%%%第1题
\item 
一随机参数\(\theta\)通过对另一随机变量\(z\)来观测,已知

\[p(z|\theta ) = \left\{ {\begin{array}{*{20}{c}}
{\theta {e^{ - \theta z}},z \ge 0,\theta  > 0}\\
{0,\theta  < 0}
\end{array}} \right.\]
假定\(\theta\)的先验密度为

\[p(\theta ) = \left\{ {\begin{array}{*{20}{c}}
{\frac{{{l^n}}}{{\Gamma (n)}}{e^{ - \theta l}}{\theta ^{n - 1}},\theta  \ge 0}\\
{0,\theta  < 0}
\end{array}} \right.\]

试求\({{\hat \theta }_{MAP}}\)与\({{\hat \theta }_{MS}}\),并求\(E\{ {[\theta  - {{\hat \theta }_{MS}}]^2}\} \)。

\textbf{解}:\\
由题目可知:
\[\begin{array}{l}
\ln p(z|\theta ) = \ln \theta  - \ln z\\
\ln p(\theta ) = \ln \left( {\frac{{{l^n}}}{{\Gamma (n)}}} \right) - \theta l + (n - 1)\ln \theta 
\end{array}\]

因此,有
\[\frac{{\partial \ln p(z|\theta )}}{{\partial \theta }} + \frac{{\partial \ln p(\theta )}}{{\partial \theta }} = \frac{1}{\theta } - z - l + \frac{{n - 1}}{\theta }\]
令其结果为0,得\({{\hat \theta }_{MAP}}\),其值为:
\[{{\hat \theta }_{MAP}} = \frac{n}{{z + l}}\]

同样,对\({{\hat \theta }_{MS}}\),有

\begin{equation*}
\begin{aligned}
{{\hat \theta }_{MS}} =& \frac{{\int_0^\infty  {\theta p(z|\theta )p(\theta )} d\theta }}{{\int_0^\infty  {p(z|\theta )p(\theta )} d\theta }}\\
=& \frac{{(n + 1)!}/{(z + l)}^{n + 2}}{n!/{(z + l)}^{n + 1}}\\
=& \frac{{n + 1}}{{z + l}}
\end{aligned}
\end{equation*}

接下来求解\(E\{ {[\theta  - {{\hat \theta }_{MS}}]^2}\} \)\\
因为
\begin{equation*}
\begin{split}
p(\theta ;z) =& p(z|\theta )p(\theta )\\
 =& \left\{ {\begin{array}{*{20}{c}}
{\frac{{{l^n}}}{{\Gamma (n)}}{\theta ^n}{e^{ - \theta (z + l)}},z \ge 0,\theta  > 0}\\
{0,\theta  < 0}
\end{array}} \right.
\end{split}
\end{equation*}

所以,有
\[\begin{aligned}
  E\{{{[\theta -{{{\hat{\theta }}}_{MS}}]}^{2}}\}
  &=\iint{\left[ {{\theta }^{2}}-2\theta \frac{n+1}{z+l}+{{\left( \frac{n+1}{z+l} \right)}^{2}} \right]p(\theta ;z)d\theta dz} \\ 
 & =\frac{{{l}^{n}}}{\Gamma (n)}\int\limits_{0}^{\infty }{\left[ \frac{(n+2)!}{{{(z+l)}^{n+3}}}-2\frac{(n+1)!\times (n+1)}{{{(z+l)}^{n+3}}}+\frac{n!\times {{(n+1)}^{2}}}{{{(z+l)}^{n+3}}} \right]dz} \\ 
 & ={{l}^{n}}\int\limits_{0}^{\infty }{\frac{n(n+1)(n+2)-2n{{(n+1)}^{2}}+n{{(n+1)}^{2}}}{{{(z+l)}^{n+3}}}dz} \\ 
 & ={{l}^{n}}\int\limits_{0}^{\infty }{\frac{n(n+1)}{{{(z+l)}^{n+3}}}dz} \\ 
 & =\frac{{{l}^{n}}n(n+1)}{-(n+2)}{{(z+l)}^{-(n+2)}}|_{0}^{\infty } \\ 
 & =\frac{n(n+1)}{{{l}^{2}}(n+2)} \\ 
\end{aligned}\]

%%%%%%%%%第2题
\item 
根据一次观测\(z\)来估计信号的参量\(\theta\)。已知\\
$$
\begin{aligned}
p(\theta)=&2exp(-2\theta), \quad \theta \ge 0\\
p(z|\theta)=&\theta exp(-z\theta), \quad \theta \ge 0,z\ge 0\\
\end{aligned}
$$

\begin{enumerate}[(1)]
\item
求估计量\({{\hat \theta }_{MAP}}(z)\)和\({{\hat \theta }_{MS}}(z)\)
\item
若\(z=2\),求对应的估计值;若\(z=4\),求对应的估计值。
\end{enumerate}

\textbf{解}:\\
\begin{enumerate}[(1)]
\item
对\({{\hat \theta }_{MS}}(z)\),根据定义计算得:
\[\begin{aligned}
{{{\hat{\theta }}}_{MS}}
& =\frac{\int_{0}^{\infty }{\theta p(z|\theta )p(\theta )}d\theta }{\int_{0}^{\infty }{p(z|\theta )p(\theta )}d\theta } \\ 
 & =\frac{\int_{0}^{\infty }{2{{\theta }^{2}}{{e}^{-\theta (z+2)}}}d\theta }{\int_{0}^{\infty }{2\theta {{e}^{-\theta (z+2)}}}d\theta } \\ 
 & =\frac{2/(z+2)^3}{1/(z+2)^2}\\ 
 & =\frac{2}{z+2} \\ 
\end{aligned}\]
对\({{\hat \theta }_{MAP}}(z)\),因为
\[\begin{aligned}
  & \ln p(z|\theta )=\ln \theta -z\theta  \\ 
 & \ln p(\theta )=\ln 2-2\theta  \\ 
\end{aligned}\]
所以,有
\[\frac{\partial \ln p(z|\theta )}{\partial \theta }+\frac{\partial \ln p(\theta )}{\partial \theta }=\frac{1}{\theta }-z-2\]
令其为0,可得:
\[{{{\hat{\theta }}}_{MAP}}=\frac{1}{z+2}\]\\

\item
当$z=2$时,\({{\hat \theta }_{MS}}(2)=1/2\),\({{\hat \theta }_{MAP}}(2)=1/4\)\\
当$z=4$时,\({{\hat \theta }_{MS}}(4)=1/3\),\({{\hat \theta }_{MAP}}(4)=1/6\)

\end{enumerate}

%%%%%%第3题
\item
给定独立观测序列$z_1,z_2,...,z_N$,具有均值$m$,方差$\sigma ^2$。
\begin{enumerate}[(1)]
\item
%%%3.1
问取样平均$$\mu = \frac{1}{N}\sum_{i=1}^N z_i$$是否为$m$的无偏估计?$\mu$的方差是什么?
\item
%%%3.2
可以找出方差的估计为$$V=\frac{1}{N}\sum_{i=1}^N[z_i-\mu]^2$$问这是否是$\sigma^2$的无偏估计?试求他的方差。
\end{enumerate}
\textbf{解}:\\
\begin{enumerate}[(1)]
\item
因为$$E[\mu]= \frac{1}{N}\sum_{i=1}^N E(z_i)=\frac{Nm}{N}=m$$
所以$\mu$是$m$的无偏估计。\\

若取$s_i=z_i-m$,则$E[s_is_j]=0,\quad i \ne j$
$$
\begin{aligned}
\text{var}(\mu)
&=E[(\mu-m)^2]\\
&=\frac{1}{N^2}E[(\sum_{i=1}^Ns_i)^2)]\\
&=\frac{1}{N^2}E[\sum_{i=1}^Ns_i^2)]\\
&=\frac{N\sigma^2}{N^2}\\
&=\frac{\sigma^2}{N}
\end{aligned}
$$
即$\mu$的方差为$\frac{\sigma^2}{N}$\\
\item
因为
$$
\begin{aligned}
E[V]
&=\frac{1}{N}\sum_{i=1}^NE[(z_i-\mu)^2]\\
&=\frac{1}{N}\sum_{i=1}^NE\{[(z_i-m)-(\mu-m)]^2\}\\
&=\frac{1}{N}\sum_{i=1}^N\{E[(z_i-m]^2 -2E[(z_i-m)(\mu-m)] +E[(\mu-m)^2]\}\\
&=\frac{1}{N}\sum_{i=1}^N\{\sigma^2-2\frac{1}{N}E[(z_i-m)^2]+\frac{\sigma^2}{N}\}\\
&=\frac{1}{N}\sum_{i=1}^N(\sigma^2-\frac{\sigma^2}{N})\\
&=\frac{N-1}{N}\sigma^2\\
&\ne \sigma^2\\
\end{aligned}
$$
所以$V=\frac{1}{N}\sum_{i=1}^N[z_i-\mu]^2$不是$\sigma^2$的无偏估计。\\

从样本的角度出发:\\
因为对于样本方差$S^2$,有$$S^2=\frac{1}{N-1}\sum_{i=1}^N(z_i-\mu)^2=\frac{N}{N-1}V$$
$$E(S^2)=\sigma^2$$
$$\text{Var}(S^2)=\frac{1}{N}\left( m^4-\frac{N-3}{N-1}\sigma^4\right)$$
所以$$E[V]=\frac{N-1}{N}E(S^2)=\frac{N-1}{N}\sigma^2$$
对于$V=\frac{1}{N}\sum_{i=1}^N[z_i-\mu]^2$,有
$$
\begin{aligned}
\text{Var}(V)
&=\text{Var}(S^2) \frac{(N-1)^2}{N^2}\\
&=\frac{(N-1)^2}{N^2} \times \frac{1}{N}\left( m^4-\frac{N-3}{N-1}\sigma^4\right) \\
&=\frac{(N-1)^2}{N^3}\left( m^4-\frac{N-3}{N-1}\sigma^4\right) \\
\end{aligned}
$$
特殊的,当$z_i$服从正态分布
$$
\begin{aligned}
\text{Var}(V)
&=\text{Var}(S^2) \frac{(N-1)^2}{N^2}\\
&=\frac{(N-1)^2}{N^2} \times \frac{2\sigma^4}{N-1} \\
&=\frac{2(N-1)\sigma^4}{N^2}\\
\end{aligned}
$$
\end{enumerate}

\item
%%%%%%%%第4题
若观测方程为$$z_k=h_k\theta + n_k \quad k=1,2,...,N$$其中$\theta$是方差为$\sigma_{\theta}^2$的零均值高斯随机变量;$n_k$是方差为$\sigma_{n}^2$的零均值高斯白噪声。
\begin{enumerate}[(1)]
\item
求$\theta$的最小均方误差估计$\hat\theta_{MS}(z)$和最大后验概率估计$\hat\theta_{MAP}(z)$,并考察其主要性能。
\item
如果$\theta$具有瑞利分布,即
\[p(\theta )=\left\{ \begin{matrix}
   \frac{\theta }{\sigma _{\theta }^{2}}\exp \left( -\frac{{{\theta }^{2}}}{2\sigma _{\theta }^{2}} \right),\theta \ge 0  \\
   0,\qquad\theta <0  \\
\end{matrix} \right.\]
求$\theta$的最大后验概率估计$\hat\theta_{MAP}(z)$
\end{enumerate}
\textbf{解}:\\
%%%这题公式好长啊,,,,不想打,,,,还好有奶茶,,,不然活不下去,,,,,
\begin{enumerate}[(1)]
\item
由题目可知,$\bm{z}=\bm{h}\theta+\bm{n}$\\
因为\\
$$
\begin{aligned}
  p(\theta |z)&=\frac{p(z|\theta )p(\theta )}{p(z)} \\ 
 p(z|\theta )&={{\left( \frac{1}{2\pi \sigma _{n}^{2}} \right)}^{\frac{N}{2}}}\exp \left( -\sum\limits_{i=1}^{N}{\frac{{{({{z}_{i}}-{{h}_{i}}\theta )}^{2}}}{2\sigma _{n}^{2}}} \right) \\ 
 p(\theta )&={{\left( \frac{1}{2\pi \sigma _{\theta }^{2}} \right)}^{\frac{1}{2}}}\exp \left( -\frac{{{\theta }^{2}}}{2\sigma _{\theta }^{2}} \right) \\ 
\end{aligned}
$$
观测可知$p(\theta|z)$也服从高斯分布,所以\\
\[\begin{aligned}
 p(\theta |z)
 & ={{K}_{1}}(z)\exp \left[ -\sum\limits_{i=1}^{N}{\frac{{{({{z}_{i}}-{{h}_{i}}\theta )}^{2}}}{2\sigma _{n}^{2}}}-\frac{{{\theta }^{2}}}{2\sigma _{\theta }^{2}} \right] \\ 
 & ={{K}_{2}}(z)\exp \left[ -\frac{{{\theta }^{2}}}{2\sigma _{\theta }^{2}}-\frac{\sum\limits_{i=1}^{N}{(-2{{z}_{i}}{{h}_{i}}\theta +{{h}_{i}}{{\theta }^{2}})}}{2\sigma _{n}^{2}} \right] \\ 
 & ={{K}_{3}}(z)\exp \left[ -\frac{\left( \sigma _{n}^{2}+\sigma _{\theta }^{2}\sum\limits_{i=1}^{N}{h_{i}^{2}} \right){{\theta }^{2}}-2\theta \sigma _{\theta }^{2}\sum\limits_{i=1}^{N}{{{z}_{i}}{{h}_{i}}}}{2\sigma _{n}^{2}\sigma _{\theta }^{2}} \right] \\ 
 & ={{K}_{3}}(z)\exp \left[ -\frac{{{\theta }^{2}}-2\theta \sigma _{\theta }^{2}\sum\limits_{i=1}^{N}{{{z}_{i}}{{h}_{i}}}/\left( \sigma _{n}^{2}+\sigma _{\theta }^{2}\sum\limits_{i=1}^{N}{h_{i}^{2}} \right)}{2\sigma _{n}^{2}\sigma _{\theta }^{2}/\left( \sigma _{n}^{2}+\sigma _{\theta }^{2}\sum\limits_{i=1}^{N}{h_{i}^{2}} \right)} \right] \\ 
\end{aligned}\]
其中$K_1(z)\ K_2(z)\ K_3(z)\ $均为与$\theta$无关的系数。\\
因此,有
\[\begin{aligned}
  {{{\hat{\theta }}}_{MAP}}\text{=}{{{\hat{\theta }}}_{MS}}&=\frac{\sigma _{\theta }^{2}\sum\limits_{i=1}^{N}{{{z}_{i}}{{h}_{i}}}}{\sigma _{n}^{2}+\sigma _{\theta }^{2}\sum\limits_{i=1}^{N}{h_{i}^{2}}} \\ 
 \operatorname{var}({{{\hat{\theta }}}_{MAP}})=\operatorname{var}({{{\hat{\theta }}}_{MS}})&=\frac{\sigma _{n}^{2}\sigma _{\theta }^{2}}{\sigma _{n}^{2}+\sigma _{\theta }^{2}\sum\limits_{i=1}^{N}{h_{i}^{2}}} \\ 
\end{aligned}\]
考察其无偏性,因为
\[E[{{{\hat{\theta }}}_{MAP}}\text{ }\!\!]\!\!\text{ =E }\!\![\!\!\text{ }{{{\hat{\theta }}}_{MS}}]=\frac{\sigma _{\theta }^{2}\sum\limits_{i=1}^{N}{E[{{z}_{i}}]{{h}_{i}}}}{\sigma _{n}^{2}+\sigma _{\theta }^{2}\sum\limits_{i=1}^{N}{h_{i}^{2}}}=0=E[\theta ]\]
所以其估计是无偏的。
考察其Cramer-Rao下界,为
\[{{\left\{ -E\left[ \frac{{{\partial }^{2}}\ln p(z;\theta )}{\partial {{\theta }^{2}}} \right] \right\}}^{-1}}=\frac{\sigma _{n}^{2}\sigma _{\theta }^{2}}{\sigma _{n}^{2}+\sigma _{\theta }^{2}\sum\limits_{i=1}^{N}{h_{i}^{2}}}=\operatorname{var}\left( {{{\hat{\theta }}}_{MAP}} \right)\]
所以其无偏估计是有效的。
\item
当$\theta$具有瑞利分布时,显然\({{{\hat{\theta }}}_{MAP}}>0\),因此
\[\frac{\partial \ln p(z\text{ }\!\!|\!\!\text{ }\theta )}{\partial \theta }\text{+}\frac{\partial \ln p(\theta )}{\partial \theta }=-\sum\limits_{i=1}^{N}{\frac{(\theta {{h}_{i}}-{{z}_{i}}){{h}_{i}}}{\sigma _{n}^{2}}}\text{+}\frac{1}{\theta }-\frac{\theta }{\sigma _{\theta }^{2}}\]
令其为0可得\({{{\hat{\theta }}}_{MAP}}\),整理得
\[\hat{\theta }_{MAP}^{2}\left( \frac{1}{\sigma _{\theta }^{2}}+\sum\limits_{i=1}^{N}{\frac{h_{i}^{2}}{\sigma _{n}^{2}}} \right)-\hat{\theta }_{MAP}^{2}\sum\limits_{i=1}^{N}{\frac{{{z}_{i}}{{h}_{i}}}{\sigma _{n}^{2}}}-1=0\]
因此可得:
\[{{{\hat{\theta }}}_{MAP}}=\frac{\sum\limits_{i=1}^{N}{\frac{{{z}_{i}}{{h}_{i}}}{\sigma _{n}^{2}}}+\sqrt{{{\left( \sum\limits_{i=1}^{N}{\frac{{{z}_{i}}{{h}_{i}}}{\sigma _{n}^{2}}} \right)}^{2}}+4\left( \frac{1}{\sigma _{\theta }^{2}}+\sum\limits_{i=1}^{N}{\frac{h_{i}^{2}}{\sigma _{n}^{2}}} \right)}}{2\left( \frac{1}{\sigma _{\theta }^{2}}+\sum\limits_{i=1}^{N}{\frac{h_{i}^{2}}{\sigma _{n}^{2}}} \right)}\]
\end{enumerate}
%%%%%%%%%做题十分钟,打题半小时%%%%%%%%%%
%%%%%%%%%注意不要打开PDF后还编译%%%%%%%%%
\end{enumerate}
\end{document}