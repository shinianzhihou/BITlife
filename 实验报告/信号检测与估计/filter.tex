%!TEX program=xelatex
\documentclass[a4paper]{article}
%packages
\usepackage[UTF8]{ctex} %中文字符
\CTEXoptions[today=old]
\usepackage[left=2.50cm, right=2.50cm, top=2.50cm, bottom=2.50cm]{geometry} %调整页边距
\usepackage{xeCJK}
\usepackage{enumerate}%使用枚举
\usepackage{mathtools}
\usepackage{titlesec}
%\renewcommand{\thesection}{\chinese{section}.}%设置section
%\titleformat{\section}{\raggedright\large\rm\bf}{\thesection .\quad}{0pt}{}[]%右对齐4号字加粗标号后面有个点
\usepackage{bm}%加粗数学公式
\usepackage{amssymb}%因为所以
\usepackage{mathrsfs} %花体字
\usepackage{amsmath}%不太花的花体字
\usepackage[colorlinks,linkcolor=black,anchorcolor=blue,citecolor=green]{hyperref}%超链接
\renewcommand{\contentsname}{\centerline{目\ 录}}
\usepackage{indentfirst}%首行缩进
\setlength{\parindent}{2em}
\usepackage{graphicx}%图片
\usepackage{listings} %代码
\usepackage{xcolor}%颜色

\begin{document}
%%%%%%%%%%%%%%%%%%%%封面%%%%%%%%%%%%%%%%%%%%%
\title{匹配滤波器的设计与研究}
\author{ 施念 1120161302}
\maketitle{}
\newpage

%%%%%%%%%%%%%%%%%%%%%目录%%%%%%%%%%%%%%%%%%%%%
\begin{center}
\tableofcontents\label{c}
\end{center}
\newpage
%%%%%%%%%%%%%%%%%%%%正文%%%%%%%%%%%%%%%%%%%%%%
%%%%%%%%%%%%%%%%%%%%%%%%%%
\section{实验目的}
\begin{enumerate}[(1) ]
\item
理解匹配滤波器的真正含义,体会匹配滤波器在实际工程中的应用,设计并实现一个简单的匹配滤波器。
\item
分别以信号种类、噪声和输入信噪比为变量,研究匹配滤波器在不同情况下的输出信噪比情况。
\end{enumerate}

%%%%%%%%%%%%%%%%%%%%%%%%%%
\section{实验原理}
%%%%%
\subsection{匹配滤波器}
在雷达发展的初期,普遍地采用信噪比作为衡量雷达接收机抗干扰性能的准则。对于一定的输入信号和噪声而言,给出输出信噪比大的系统一般说来要比给出信噪比小的系统为好。匹配滤波器就是这样一种线性滤波器,在所有的线性滤波器中,在它的输出端能给最大的信号噪声功率比。
%%%%%
\subsection{设计理论分析}
\subsubsection{最大输出信噪比的线性滤波器}
假设在滤波器的接受端接收到的是信号和加性噪声,如图1所示,经过线性滤波器在输出端会得到信号和噪声分别经过滤波器得到输出的组合(线性)。
\begin{figure}[ht]
\centering
\includegraphics[scale=0.7]{imgs/linearfilter.png}
\caption{线性滤波器结构}
\end{figure}

假定滤波器的的输出信号的峰值出现在$t = {t_0}$时刻,记为${s_0}({t_0})$,它是一个确定量。
$${s_0}({t_0}) = \frac{1}{{2\pi }}\int_{ - \infty }^\infty  {S(\omega )H(\omega ){e^{j\omega {t_0}}}{\rm{d}}\omega } $$


此时,输出信号的瞬时功率为
$${\left| {{s_0}({t_0})} \right|^2} = {\left| {\frac{1}{{2\pi }}\int_{ - \infty }^\infty  {S(\omega )H(\omega ){e^{j\omega {t_0}}}{\rm{d}}\omega } } \right|^2}$$

此时瞬时功率信噪比为
$$\rho  = \frac{{{{\left| {{s_0}({t_0})} \right|}^2}}}{{E[n_0^2(t)]}}= \frac{{{{\left| {\frac{1}{{2\pi }}\int_{ - \infty }^\infty  {S(\omega )H(\omega ){e^{j\omega {t_0}}}{\rm{d}}\omega } } \right|}^2}}}{{\frac{{{N_0}}}{{4\pi }}\int_{ - \infty }^\infty  {{{\left| {H(\omega )} \right|}^2}{\rm{d}}\omega } }}$$

在观察回波时,为达到最大输出信噪比,根据函数内积的定义
$$<f,g>={\int_R f \cdot g^* dr}$$
$$|<f,g>|^2 \leq\, <f,f><g,g>$$

我们可得到若在\(\tau\)延时时刻观察,则与其对应的的匹配滤波器为
$${H}(\omega ) = k{S^ * }(\omega ){e^{ - j\omega {\tau}}}$$
$$h(t)=ks(\tau-t)$$

将\({H}(\omega )\)代入信噪比$\rho$的表达式中我们可以进一步推倒出匹配滤波器在其输出能给的最大信噪比$$\rho=\frac{2E}{N_0}$$

从表达式我们可以看出,最大输出信噪比仅与输入信号能量 和白噪声的功率谱密度 有关,而与输入信号的波形、噪声的分布律无关。输入信号的波形只影响匹配滤波器传输函数的形状。只要各种信号的能量相同,白噪声的功率谱密度相同,与它们相应的匹配滤波器的输出信噪比就都一样。换句话说,在同样的噪声干扰条件下,只有增加信号的能量,才能提高匹配滤波器的检测能力。

%%%%
\subsubsection{物理可实现性}
我们都知道,一个系统若是物理可实现的,在输入信号为实函数时,其传函需要满足
$${H_m}(\omega ) = \int_0^\infty  {{e^{ - j\omega t}}} {\rm{d}}t\frac{1}{{2\pi }}\int_{ - \infty }^\infty  {S( - \omega ){e^{ - j\omega ({\tau} - t)}}} {\rm{d}}\omega  = \int_0^\infty  {s({\tau} - t){e^{ - j\omega t}}} {\rm{d}}t$$

对应的脉冲响应为
$${h_m}(t) = \left\{ {\begin{array}{*{20}{c}}
{s({\tau} - t)}&{t \ge 0}\\
0&{t < 0}
\end{array}} \right.$$
%%%%
\subsubsection{观察时延$\tau$}
从上述的物理可实现性要求我们可以得到$t_0$和$\tau$的关系为$$\tau \ge t_{0}$$

从信号接收的角度很好理解----我们只有在完全接收信号的前提下才能达到理论上的最大信噪比,此时系统才是物理可实现的。特殊的,当$\tau = t_{0},k=1$时,匹配滤波器的脉冲响应与信号关于$t=t_0/2$对称。


%%%%%%%%%%%%%%%%%%%%%%%%%%
\section{实验内容}
%%%%
\subsection{设计}
\subsubsection{白噪声情况下的匹配滤波器}
以正弦信号$s_1=Acos(2\pi f_0t)$作为原始信号,加上白噪声$n_i=1*randn(1,N)$一起作为输入,通过时域为$h=ks(\tau-t)$  频域为${H}(\omega ) = k{S^ * }(\omega ){e^{ - j\omega {\tau}}}$的匹配滤波器,观察输出的时域和幅频响应,结果如下图($f_0=10Hz,fs=1kHz,A=1,N=1000$)所示。
\newpage


\begin{figure}[htbp]
\centering
\includegraphics[scale=0.6]{imgs/filter0.png}
\caption{匹配滤波器}
\end{figure}

通过计算我们得到$SNR_i=-2.917dB\quad SNR_o=31.7119dB$,将匹配滤波器理解为相关运算,尽管输入信号混有噪声,但是经过匹配滤波器后,因为噪声和信号不相关,所以输出端噪声就被''滤去''。输出信号的频率和输入信号一样,包络为对称的三角波形,顶点位于\(t=\tau_0/2=t_0\)处,振荡的瞬时峰值也在这一点,同时在这点达到了最大输出信噪比。



%%%%%%%%%%
\subsection{研究}
\subsubsection{延时$\tau$对输出的影响}
在上面的实验中我们发现,当\(\tau \)取信号的长度$t_0$时,输出信号刚好在$\tau=t_0$取得理论上的最大输出信噪比。

因为滤波器要求是物理可实现性的,因此当我们将延时\(\tau \)依次取$t_0/2$,$t_0$和$3t_0/2$时,匹配滤波器的冲激响应如下图所示。

\begin{figure}[htbp]
\centering
\includegraphics[scale=0.6]{imgs/filter1.png}
\caption{时延$\tau$不同的匹配滤波器冲激响应}
\end{figure}

将输入信号依次通过这些延时不同的滤波器,得到输出如下图所示。
\newpage


\begin{figure}[htbp]
\centering
\includegraphics[scale=0.6]{imgs/filter2.png}
\caption{时延$\tau$不同的匹配滤波器输出}
\end{figure}


由上图和下表可知,{\bfseries 只有在时延$\tau \ge t_0$时输出信号才会到达最大输出信噪比。从信号能量接收的角度来说,只有在接收机完全将信号接收才有可能达到最大信噪比($\rho=2E/N_0$)}。
\begin{table}[h]
\centering
\caption{不同信噪比输入信号输出信噪比对比}
\begin{tabular}[h]{ccc}
\hline
时延$\tau$& 输入信噪比(dB)& 输出信噪比(dB)\\
\hline
$0.25t_s$ & -2.7639 &25.3063\\
$0.50t_s$ & -2.7639 &29.0912\\
$0.75t_s$ & -2.7639 &30.7183\\
$1.00t_s$ & -2.7639 &31.8673\\
$1.25t_s$ & -2.7639 &31.8673\\
$1.50t_s$ & -2.7639 &31.8673\\
\hline
\end{tabular}
\end{table}


\subsubsection{以信号种类为变量}
在其他条件一定的情况下,依次选取不同波形的函数作为信号(如下图所示),加噪声后作为滤波器的接收信号。

\begin{figure}[htbp]
\centering
\includegraphics[scale=0.5]{imgs/s4_7.png}
\caption{不同输入信号(未加噪声)}
\end{figure}
%%%%%
\newpage
我们将这些信号加上噪声后作为滤波器的输入,通过仿真,我们得到下表(N=1000,A=2):

%表1
\begin{table}[htbp]
\centering
\caption{不同波形输入信号输出信噪比对比}
\begin{tabular}[htbp]{cccc}
\hline
输入信号&信号函数(MATLAB表达式)& 输入信噪比(dB)& 输出信噪比\\
\hline
$s_4$	&$A*ones(1,N)$					&6.0873&46.8246 \\
$s_5$	&$A*repmat([1\ -1],1,N/2)$			&6.0873&43.6760\\
$s_6$	&$A*repmat([1\ 1\ 1\ -1],1,N/4)$	&6.0873&40.4825\\
$s_7$	&$A*repmat([0\ 0\ 0\ 2],1,N/4)$		&6.0873&40.1935\\
\hline
\end{tabular}
\end{table}

结合上面的分析,通过简单计算我们可以得知这些新号虽然波形不同,但是能量是相同的。因此我们可以得出结论,{\bfseries 匹配滤波器的最大输出信噪比和信号的形状无关,与信号的能量有关}。这个结论与我们之前的理论推导得出的$\rho = 2E/N_0$相符合。
%%%%%%%%%%%%%
\subsubsection{以输入信噪比为变量}
在其他条件一定的情况下,如下图所示,只改变信号的幅值,以此来改变输入信号的信噪比,观察输出信号的波形以及信噪比。
\begin{figure}[h]
\centering
\includegraphics[scale=0.5]{imgs/s1_3.png}
\caption{信噪比不同的输入信号(未加噪声)}
\end{figure}

将这些信号加上噪声之后通过匹配滤波器,得到输出信号的波形如下图所示.
\newpage

%%%%图4
\begin{figure}[h]
\centering
\includegraphics[scale=0.5]{imgs/y1_3.png}
\caption{输出信号}
\end{figure}

通过计算,得到信噪比如下表(其中f0=20,A=1)所示:


%%%%表2
\begin{table}[h]
\centering
\caption{不同信噪比输入信号输出信噪比对比}
\begin{tabular}[h]{cccc}
\hline
输入信号&信号函数(MATLAB表达式)& 输入信噪比(dB)& 输出信噪比(dB)\\
\hline
$s_1$&$A*cos(2*pi*f0*t)$&-2.7549&33.1134 \\
$s_2$&$2*A*cos(2*pi*f0*t)$&3.2657&39.0783\\
$s_3$&$4*A*cos(2*pi*f0*t)$&9.2863&45.0709\\
\hline
\end{tabular}
\end{table}
%%%%分析

通过分析我们可以看出,在信号幅度$*2$之后,输入信噪比和输出信噪比(dB)都$+6dB\ (+10log_{10} 2^2)$。

从仿真的角度分析,在仿真中我们直接取最大输出信噪比$\rho_{max} = |s_{o}(t_0)|/E[n_o^2(t)]$,当输入信号的幅度$*2$时,信号通过匹配滤波器(和经过延时的信号本身做相关运算),在$t_0$时刻的输出变为原信号的$2^2$倍。

从理论分析的角度来看,结合上面提到的$\rho = 2E/N_0$,在信号幅度$*2$时,信号的能量$*2^2$,用dB表示就是$+6dB$

最后我们可以得出结论,{\bfseries 输入信号的幅度每增加一倍,输出信号的最大信噪比增加6dB。并且输入信号的信噪比增加xdB,输出信号的信噪比也增加xdB}。
%%%%%%%%%%%%%%%%%%%%%%%%%%

%%%%%%%%%%%%%%%%%%%%%%%%%%
\section{总结与分析}
\subsection{问题总结}
\begin{enumerate}[(1)]
\item
在进行模拟仿真时,为了更好的观察系统在时域的特性,要注意将时间轴t和延时$\tau$结合起来,使滤波器在时域的单位冲激响应随着时延$\tau$的改变而改变,从而改变滤波器的输出。除此之外,为了更好的观察,可以适当将纵轴用ylim函数固定,方便进行对比。

\item 
在进行以信噪比为变量研究输出信号比的时候,只考虑了输入原信号通过匹配滤波器,没有考虑输入噪声和输出噪声的不同(长度、形状),仍然选用输入噪声的统计特性作为输出噪声的统计特性,因此在仿真时,输出信噪比递增的步长是12dB,而不是6dB,后来经过分析和验证将错误改正。

\end{enumerate}

\subsection{实验总结}
此次实验探究,不仅让我更加熟悉了书本上跟匹配滤波器相关公式的推倒,从时域和频域理解了匹配滤波器的特点,更让我实际上明白了什么叫做匹配滤波----不仅可以从相关接收的角度来理解,还可以能量的角度来理解。

在探究过程中,从延时$\tau$、信号种类、信号能量和输入信噪比几个方面研究了什么因素能改变最大输出信噪比,加深了对''最大输出信噪比只和信号的能量和噪声的功率谱密度有关''的理解。除此之外,在进行仿真实验探究的过程中,更加熟悉了$Latex$的编程排版以及MATLAB一些较好用的函数。


\end{document}}