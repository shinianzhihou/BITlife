%!TEX program=xelatex
\documentclass{article}
%packages
\usepackage[UTF8]{ctex} %中文字符
\CTEXoptions[today=old]
\usepackage[left=2.50cm, right=2.50cm, top=2.50cm, bottom=2.50cm]{geometry} %调整页边距
\usepackage{xeCJK}
\usepackage{enumerate}%使用枚举
\usepackage{mathtools}
\usepackage{titlesec}
\renewcommand{\thesection}{\chinese{section}}%设置section
\usepackage{bm}%加粗数学公式
\usepackage{amssymb}%因为所以
\usepackage{mathrsfs} %花体字
\usepackage{amsmath}%不太花的花体字
% MATLAB代码插入包
\usepackage{listings}
\usepackage[framed,numbered,autolinebreaks,useliterate]{mcode}

\begin{document}
\title{《信号检测与估计》作业二\footnote{习题2.1\quad2.2\quad2.3\quad2.4}}
\author{施念  1120161302}
\date{April 12, 2019}
\maketitle{}
%\begin{abstract}%摘要部分
%\end{abstract}
%\tableofcontents%生成目录
\section{第2章\ 检测基本理论}
\begin{enumerate}[1.]
%%%%%%%%%%第1题
\item 
根据N个独立样本,设计一个似然比检验,对下列假设进行选择
\begin{equation*}
\begin{split}
H_0:z(t)&=n(t)\\
H_1:z(t)&=1+n(t)\\
H_2:z(t)&=-1+n(t)\\
\end{split}
\end{equation*}
其中n(t)是零均值、方差$\sigma^2$的高斯过程。假定各假设的先验概率相等,正确判断无代价,任何错误的代价相同。证明检验统计量可选为$l(z)=\frac{1}{N}\sum_{i=1}^{N}z_i$。求$l(z)$的判决区域。

\textbf{解}:\\
根据各假设的先验概率相等以及代价因子的特性,三元假设检验可按照最大似然准则来设计。\\
N维观测矢量的似然函数为
\begin{equation*}
\begin{split}
p(\mathbf{z}|H_i)&=\prod_{k=1}^{N} P(z_k|H_i)\\
&=\left(  \frac{1}{2\pi\sigma _n^2} \right)^{\frac{N}{2}}exp\left[ -\sum_{k=1}^{N} \frac{(z_k-s_i)^2}{2\sigma_n^2}  \right],s_i(i)=0(0),1(1),-1(2)\\
\end{split}
\end{equation*}
似然比为
\[\Lambda (x) = \frac{{p(x|{H_i})}}{{p(x|{H_j})}}\mathop{\gtrless}\limits_{H_i}^{H_j}1\]
因此三元假设判决规则等价于选择最大的
\[\frac{{2{s_i}}}{N}\sum\limits_{k = 1}^N {{z_k} - s_i^2\qquad i = 0,1,2} \]
故对检验统计量$\hat z = l(z)=\frac{1}{N}\sum_{i=1}^{N}z_i$,判决区域为:
\begin{center}
$l(z)\le-0.5$,判断$H_2$成立\\
$-0.5\le l(z)\le0.5$,判断$H_0$成立\\
$l(z)\ge0.5$,判断$H_1$成立\\
\end{center}


%%%%%%%%%第2题
\item 
根据一次观测,用极大极小化检
验来对下面两个假设做出判断。
\begin{equation*}
\begin{split}
H_0:z(t)&=1+n(t)\\
H_1:z(t)&=n(t)\\
\end{split}
\end{equation*}
假定n(t)是具有零均值和功率$\sigma^2$的高斯过程,以及$C_{00}=C_{11}=0,\ C_{01}=C_{10}=1$。根据观测结果给出的门限是什么?答案意味着每一个假设的先验概率是多少?\\
\textbf{解}:\\
因为$C_{00}=C_{11}=0,\ C_{01}=C_{10}=1$,所以对代价函数有
$$C(P_1,P_{1g})=(1-P_1)P_F(P_{1g})+P_M(P_{1g})P_1$$
求导并令结果为0得
$$P_F(P_{1g})=P_M(P_{1g})$$
由题目可知,当$P_F(P_{1g})=P_M(P_{1g})$时,门限$\lambda=(1+0)/2=0.5$。
此时意味着每一个假设的先验概率相等都为$1/2$


%%%%%%第3题
\item
若题2.2假定$C_{10}=3,C_{01}=6$。
\begin{enumerate}[(1)]
\item
%%%3.1
每个假设的先验概率为何值时达到最大的可能代价?\\
\textbf{解}:\\
根据2.2,可知,当$C_{10}=3,C_{01}=6$时,欲使代价最小,有
$$P_F(P_{1g})=2P_M(P_{1g})$$
即
\[P({H_1}|{H_0}) = 2P({H_0}|{H_1})\]
结合
\[P({H_1}|{H_0}) = \frac{{P({H_0}|{H_1})P({H_1})}}{{P({H_0})}}\]
\[P({H_1}) + P({H_0}) = 1\]
可知先验概率为
\[\left\{ \begin{array}{l}
P({H_1}) = \frac{2}{3}\\
P({H_0}) = \frac{1}{3}
\end{array} \right.\]
此时达到最大的可能代价
\item
%%%3.2
根据一次观测的判决区域如何?

根据上式可知
\[\frac{{p(x|{H_1})}}{{p(x|{H_0})}}\mathop{\gtrless} \limits_{{H_0}}^{{H_1}} \frac{{({C_{10}} - {C_{00}})P({H_0})}}{{({C_{01}} - {C_{11}})P({H_1})}} = \frac{{3 \times \frac{1}{3}}}{{6 \times \frac{2}{3}}} = \frac{1}{4}\]
因为
\[\frac{{p(x|{H_1})}}{{p(x|{H_0})}}{\rm{ = }}{{\rm{e}}^{[ - \frac{{{x^2} - {{(x - 1)}^2}}}{{2{\sigma ^2}}}]}} = {e^{( - 2x + 1)/(2{\sigma ^2})}}\]
所以,有
\[{e^{( - 2x + 1)/(2{\sigma ^2})}}\mathop{\gtrless}\limits_{{H_0}}^{{H_1}} \frac{1}{4}\]
化简得
\[x\mathop{\gtrless} \limits_{{H_1}}^{{H_0}} \frac{1}{2} + {\sigma ^2}\ln4\]
判决区域为(z为一次观测量):
\begin{center}
$z\le\frac{1}{2} + {\sigma ^2}\ln 4$,判断$H_1$成立\\
$z\ge\frac{1}{2} + {\sigma ^2}\ln 4$,判断$H_0$成立\\
\end{center}

\end{enumerate}

\item
%%%%%%%%第4题
证明二元假设检验贝叶斯平均代价\(\bar C\)可表示为
\[\begin{array}{l}
\bar C = {C_{00}} + ({C_{10}} - {C_{00}})P({H_1}|{H_0}) + P({H_1})[({C_{11}} - {C_{00}}) + \\
\qquad({C_{01}} - {C_{11}})P({H_0}|{H_1}) - ({C_{10}} - {C_{00}})P({H_1}|{H_0})]\\
\end{array}\]
\textbf{解}:\\
对于二元假设检验,有
\[P({H_1}) = 1 - P({H_0})\qquad P({H_0}) = 1 - P({H_1})\]
\[P({H_0}|{H_0}){\rm{ = 1}} - P({H_1}|{H_0}){\rm{ = }}1 - {P_F}\]
\[P({H_1}|{H_1}){\rm{ = 1}} - P({H_0}|{H_1}){\rm{ = }}1 - {P_M}\]
所以对贝叶斯平均代价\(\bar C\),有
\[\begin{array}{l}
{\bar C}
 = {C_{00}}P({H_0})P({H_0}|{H_0}) + {C_{01}}P({H_1})P({H_0}|{H_1}) + \\
\qquad{C_{10}}P({H_0})P({H_1}|{H_0}) + {C_{11}}P({H_1})P({H_1}|{H_1})\\
 = {C_{00}}[1 - P({H_1})](1 - {P_F}) + {C_{01}}P({H_1}){P_M} + \\
\qquad{C_{10}}[1 - P({H_1})]{P_F} + {C_{11}}P({H_1})(1 - {P_M})\\
 = {C_{00}} + ({C_{10}} - {C_{00}}){P_F} + P({H_1})({P_F}{C_{00}} - {C_{00}}\\
 \qquad + {C_{01}}{P_M} - {C_{10}}{P_F} + {C_{11}} - {C_{11}}{P_M})\\
 = {C_{00}} + ({C_{10}} - {C_{00}}){P_F} + P({H_1})[({C_{11}} - {C_{00}}) \\
 \qquad+ ({C_{01}} - {C_{11}}){P_M} - ({C_{10}} - {C_{00}}){P_F}]\\
 = {C_{00}} + ({C_{10}} - {C_{00}})P({H_1}|{H_0}) + P({H_1})[({C_{11}} - {C_{00}}) + \\
 \qquad({C_{01}} - {C_{11}})P({H_0}|{H_1}) - ({C_{10}} - {C_{00}})P({H_1}|{H_0})]
\end{array}\]
证毕


%%%%%%%%%做题十分钟,打题半小时%%%%%%%%%%
%%%%%%%%%注意不要打开PDF后还编译%%%%%%%%%
\end{enumerate}
\end{document}