%!TEX program=xelatex
\documentclass{article}
%packages
\usepackage[UTF8]{ctex} %中文字符
\CTEXoptions[today=old]
\usepackage[left=2.50cm, right=2.50cm, top=2.50cm, bottom=2.50cm]{geometry} %调整页边距
\usepackage{xeCJK}
\usepackage{enumerate}%使用枚举
\usepackage{mathtools}
\usepackage{titlesec}
\renewcommand{\thesection}{\chinese{section}}%设置section
\usepackage{bm}%加粗数学公式
\usepackage{amssymb}%因为所以
\usepackage{mathrsfs} %花体字
\usepackage{amsmath}%不太花的花体字
% MATLAB代码插入包
\usepackage{listings}
\usepackage[framed,numbered,autolinebreaks,useliterate]{mcode}

\begin{document}
\title{《信号检测与估计》作业四 \footnote{习题2.27\quad2.28\quad3.1\quad3.2\quad3.5}}
\author{施念  1120161302}
\date{}
\maketitle{}
%\begin{abstract}%摘要部分
%\end{abstract}
%\tableofcontents%生成目录
\section{第5章\ 噪声中的信号处理}
\begin{enumerate}[1.]
\item
在功率谱密度$G_n(\omega)=\frac{N_0}{2}$的加性高斯白噪声$n(t)$背景中,假设$H_j$下的接受信号为 \[
	H_j:z(t)=s_j(t)+n(t),\quad0\le t\le 
\]
其中,信号 $ s_j(t) $的能量为 \[
	E_j=\int_0^Ts_j^2(t)dt	
\]
采用正交级数展开法时,先建立N维似然比检验 $ \Lambda[z_N] $,再取 $ N \rightarrow \infty $的极限获得$ \Lambda[z_N] $。如果在获得前N个系数 $ z_k $的联合概率密度函数(似然函数) \(p(r_n|H_j)\)后,先取 $ N \rightarrow \infty $的极限,求得似然函数 \(p(z(t)|H_j)\),再建立似然比检验,结果是一样的。请证明似然函数\(p(z(t)|H_j)\)可以表示为\[
	p[z(t)|H_j]=Fexp\left\{-\frac{1}{N_0}\int_0^T[z(t)-s_j(t)]^2dt\right \}
\]
其中\[
	F=\underset{N \rightarrow +\infty }{lim}\left(\frac{1}{\pi N_0}\right)^{N/2}
\]


\textbf{证明:}\\
设一组正交函数集为 $ \{f_k(t)\} $,则在$ \{f_k(t)\} $下,有 \[
	\begin{aligned}
	&z(t)=\underset{N \rightarrow +\infty}{lim}\sum^N_{i=1}z_if_i(t)\\
	&s(t)=\underset{N \rightarrow +\infty}{lim}\sum^N_{i=1}s_if_i(t)\\
	&n(t)=\underset{N \rightarrow +\infty}{lim}\sum^N_{i=1}n_if_i(t)\\
	&\int_0^Tf_k(t)f_l(t)dt= \delta_{k-l}\\
	\end{aligned}
\]
因此,有 \[
	\begin{aligned}
	z_k = &\int_0^Tz(t)f_k(t)dt\\
	\mathbb{E}[z_k|H_j]=&s_{jk}\\
	\text{Var}[z_k|H_j]=&\frac{N_0}{2}\\
	\int_0^Tz^2(t)dt=& \underset{N \rightarrow +\infty}{lim}\sum_{i=1}^{N}z_i^2\\
	\int_0^Tz(t)s_j(t)dt=& \underset{N \rightarrow +\infty}{lim}\sum_{i=1}^{N}z_is_{ji}\\
	\int_0^Ts_j^2(t)dt=& \underset{N \rightarrow +\infty}{lim}\sum_{i=1}^{N}s_i^2\\
	\end{aligned}
\]	
对于似然函数\(p(z(t)|H_j)\),有\[
	\begin{aligned}
	p(z(t)|H_j) &=\underset{N \rightarrow +\infty}{lim}p(z_1,z_2,\ldots,z_N|H_j)\\
	&=\left( \frac{1}{\pi N_0} \right)^{N/2}exp\left[-\frac{1}{N_0}\underset{N \rightarrow +\infty}{lim}\sum_{i=0}^{N}(z_k-s_{jk})^2\right]\\
	&=\left( \frac{1}{\pi N_0} \right)^{N/2}exp\left[-\frac{1}{N_0}\underset{N \rightarrow +\infty}{lim}\sum_{i=0}^{N}(z_k^2-2z_ks_{jk}+s^2_{jk})\right]\\
	&=\left( \frac{1}{\pi N_0} \right)^{N/2}exp\left\{-\frac{1}{N_0}\int_{0}^{T}[z^2(t)-2z(t)s_{j}(t)+s^2_{j}(t)]dt\right\}\\
	&=\left( \frac{1}{\pi N_0} \right)^{N/2}exp\left\{-\frac{1}{N_0}\int_0^T[z(t)-s_j(t)]^2dt\right\}\\
	&=Fexp\left\{-\frac{1}{N_0}\int_0^T[z(t)-s_j(t)]^2dt\right \}
	\end{aligned}
\]
证毕



\item
如果二元通信系统在两个假设下的接收信号分别为
\[
	H_0:z(t)=B\cos (\omega_2 t+\theta)+n(t),\quad 0\le t \le T
\]
\[
	H_1:z(t)=A\cos (\omega_1 t) + B\cos (\omega_2 t+\theta)+n(t),\quad 0\le t \le T
\]

式中,$ A,B,\omega_1,\omega_2 $和$\theta$均为已知的常数。噪声$n(t)$是功率谱密度$ G_n(\omega)=\frac{N_0}{2} $的高斯白噪声。设计似然比门限为$ \lambda_0 $的最佳检测系统。信号$B\cos (\omega_2 t+\theta)$对接收机性能有什么影响?

\textbf{解:}
设\[
		\begin{aligned}
		&s_0(t)=B\cos (\omega_2 t+\theta)\\
		&s_1(t)=A\cos (\omega_1 t) + B\cos (\omega_2 t+\theta)\\
		&\int_0^Ts_0^2(t)dt=E_0\\
		&\int_0^Ts_1^2(t)dt=E_1\\
		&\alpha(t)=A\cos (\omega_1 t)\\
		&\beta(t)=B\cos (\omega_2 t+\theta)\\
		\end{aligned}
\]
因此,似然比 \[
	\Lambda(z)=\frac{p[z(t)|H_1]}{	p[z(t)|H_0]}=exp\{-\frac{1}{N_0}\int_0^T[-2z(t)\alpha(t)+\alpha^2(t)+2\alpha(t)\beta(t)]dt\}\mathop{\gtrless}\limits_{H_0}^{H_1}\lambda_0
\]	
整理后可得:\[
		\int_0^T\alpha(t)z(t)dt\mathop{\gtrless}\limits_{H_0}^{H_1}\frac{N_0}{2}\ln\lambda_0+\frac{1}{2}(E_1-E_0)
\]

即对输入$z(t)$信号,将其与$ s_1(t)-s_0(t)=\alpha(t) $信号进行相乘后积分,每隔T时刻进行抽样判决,判决门限为 $ 	\frac{N_0}{2}\ln\lambda_0+\frac{1}{2}(E_1-E_0) $,大于判决门限判为发送$ s_1(t) $,否则判为$ s_0(t) $。(相关接收机)\\

讨论信号$B\cos (\omega_2 t+\theta)$对接收机性能有什么影响:

设在正交函数集中,$ f_1(t)=K[s_1(t)-s_0(t)]=\sqrt{2}\alpha(t)/A $,则有 \[
	\begin{aligned}
	l=\int_{0}^T \alpha(t)z(t)dt=&\frac{A}{\sqrt{2}}\int_{0}^T f_1(t)z(t)dt=\frac{Az_1}{\sqrt{2}}\\
	z(t)|H_0=&\underset{N\rightarrow +\infty}{\lim}\sum_{k=1}^{N}z_kf_k(t)=(s_{01}+n_{1})f_1(t)+M\\
	z(t)|H_1=&(s_{11}+n_{1})f_1(t)+M\\
	\end{aligned}	
\]
上式中,\[
	\begin{aligned}
	M=&\sum_{2}^{+\infty}z_kf_k(t) \\
	s_{11}=&\int_0^Ts_1(t)f_1(t)dt = \frac{\sqrt{2}}{A}\int_0^T[\alpha(t)+\beta(t)]\alpha(t)dt\\
	s_{01}=&\int_0^Ts_0(t)f_1(t)dt = \frac{\sqrt{2}}{A}\int_0^T\alpha(t)\beta(t)dt\\
	\end{aligned}
\]

因此,有: \[
	\begin{aligned}
	l|H_0=&\frac{Az_1}{\sqrt{2}}=\frac{A(s_{01}+n_1)}{\sqrt{2}}\\
	l|H_1=&\frac{Az_1}{\sqrt{2}}=\frac{A(s_{11}+n_1)}{\sqrt{2}}\\
	\end{aligned}
\]
即: \[
	\begin{aligned}
	l|H_0& \sim N\left( \frac{As_{01}}{\sqrt{2}},\frac{N_0}{4} \right)\\
	l|H_1& \sim N\left( \frac{As_{11}}{\sqrt{2}},\frac{N_0}{4} \right)\\
	\end{aligned}
\]
因此信噪比(偏移系数) $ d^2 $为: \[
	\begin{aligned}
	d^2=&\frac{[E(l|H_1)-E(l|H_0)]^2}{\text{var}(l|H_0)}\\
	   =&\frac{[A(s_{11}-s_{01})]^2/2}{N_0/4}\\
	   =&\frac{[\int_0^T\alpha^2(t)dt]^2}{N_0/4}\\
	   =&\frac{A^2T}{N_0}\\
	\end{aligned}
\]
与 $ \beta(t)=B\cos(w_2t+\theta) $无关,因此其对接收机性能并没有什么影响。
\\

从“两个高斯分布的信号根据门限来判别的过程”这个角度分析,经过计算(和上面基本相同)发现,门限和两个高斯分布的均值之间的距离均与$ \beta(t) $无关,因此在判决(积分)时得到的 $ P(H_1|H_1)、P(H_0|H_0) $不受$ \beta(t) $的影响。从完全统计量的角度来看,观测量在$ s_1(t)-s_0(t)=\alpha(t) $这个向量上的投影决定了判决的结果,与$ \beta(t) $无关。


\item
在高斯白噪声中检测像$ \sqrt{\frac{2E_s}{T}}\sin (\omega_0t+\theta) $这样一类确知信号时,其中相位 $ \theta $是已知的,但不一定为0.假设为雷达类型的问题,则 \[
	\begin{aligned}
	&H_0:z(t)=n(t),& 0\le t\le T\\
	&H_1:z(t)=\sqrt{\frac{2E_s}{T}}\sin(\omega_0t+\theta)+n(t),&0\le t\le T\\
	\end{aligned}
\]
其中, $ E_s $为信号的能量, $ \omega_0 $和 $ \theta $已知;噪声 $ n(t) $是功率谱密度 $ G_n(\omega)=\frac{N_0}{2} $的高斯白噪声。

\begin{enumerate}[(1)]
\item
如果在相关运算中,把相位 $ \theta $作为0来处理,但实际接收信号的 $ \theta $可能不等于0,求作为 $ \theta $函数的检测概率,并把它同 $ \theta=0 $的情况作比较。
\item
证明检测概率可能小于虚警概率,这取决于 $ \theta $的数值。
\end{enumerate}

\textbf{解:}

\begin{enumerate}[(1)]
\item
在接收信号时,判决规则为: \[
	l(z)=\int_0^Ts(t)z(t)dt\mathop{\gtrless}\limits_{H_0}^{H_1}\frac{N_0T}{2}\ln\lambda_0+\frac{1}{2}E_s=\gamma
\]
其中,发射信号$ s(t) = \sqrt{\frac{2E_s}{T}}\sin(\omega_0t)$(无相位延迟)。

在正交函数集$ \{f_k(t)\} $上,取$ f_1(t)=\frac{s(t)}{\sqrt{E_s}} $,则有 \[
	\begin{aligned}
	l(z)& = \int_0^Ts(t)z(t)dt\\
		& = \int_0^T\sqrt{E_s}f_1(t)z(t)dt\\
		& = \sqrt{E_s}z_1\\
	\end{aligned}
\]
因此,有 \[
	\begin{aligned}
	l|H_1 &= \sqrt{E_s}(s_1+n_1)\\
		  &= \int_0^Ts(t)s(t)\cos\theta dt + \sqrt{E_s}n_1\\
		  &= E_s\cos\theta + \sqrt{E_s}n_1\\
		  &\sim N(E_s\cos\theta,N_0E_s/2)\\
	l|H_0 &=\sqrt{E_s}n_1\\
		  &\sim N(0,N_0E_s/2)\\
	\end{aligned}
\]

因此检测概率为: \[
	P_D=P(H_1|H_1)=\int_{\gamma}^{+\infty}p(l|H_1)dl=Q\left(\frac{\gamma-E_s\cos\theta}{\sqrt{N_0E_s/2}}\right)
\]
特殊的,当 $ \theta=0 $时\[
	P_D=Q\left(\frac{\gamma-E_s}{\sqrt{N_0E_s/2}}\right)
\]

\item
由(1)可知,虚警概率为 \[
	P_F=P(H_1|H_0)=\int_{\gamma}^{+\infty}p(l|H_0)dl=Q\left(\frac{\gamma}{\sqrt{N_0E_s/2}}\right)
\]
当 $ P_D<P_F $时, \[
	\gamma-E_s\cos\theta<\gamma
\]
即在 \[
	\cos\theta >0 
\]
$$ 2k\pi -\frac{\pi}{2}<\theta <2k\pi +\frac{\pi}{2},\quad k=0,\pm 1,\pm2,\ldots,\pm N $$
检测概率小于虚警概率,这取决于 $ \theta $。
\end{enumerate}

\item
考虑在高斯噪声背景中检测高斯信号的问题。设 \[
	\begin{aligned}
	&H_0:z(t)=n(t),& 0\le t\le T\\
	&H_1:z(t)=s(t)+n(t),&0\le t\le T\\
	\end{aligned}
\]
其中 $ n(t) $和 $ s(t) $分别是零均值的高斯噪声和高斯信号,其带宽限于 $ |\omega|<\Omega=2\pi B $,功率谱密度分别为 $ N_0/2 $ 和 $ S_0/2 $。假设以 $ \pi/\Omega $的间隔取 $ 2BT $个样本的方式进行统计检测,试求似然比检测系统。

\textbf{解:}
因为采样频率 \[
	f_s = \frac{1}{T_s}=\frac{\Omega}{\pi}=2B
\]
所以采样所得的样本之间相互独立。\\
对第k次采样,有 \[
	\begin{aligned}
	p(z_k|H_0)&=\left(\frac{1}{2\pi N_0B}\right)^{\frac{1}{2}}exp\left( -\frac{1}{2N_0B}z_k^2 \right)\\
	p(z_k|H_1)&=\left[\frac{1}{2\pi (N_0+S_0)B}\right]^{\frac{1}{2}}exp\left[ -\frac{1}{2(N_0+S_0)B}z_k^2 \right]\\
	\end{aligned}
\]
设似然比检测门限为 $ \lambda ,\quad N=2BT,$则 \[
	\begin{aligned}
	p(z|H_0)&=\left(\frac{1}{2\pi N_0B}\right)^{\frac{N}{2}}exp\left( -\frac{1}{2N_0B}\sum_{k=1}^{N}z_k^2 \right)\\
	p(z|H_1)&=\left[\frac{1}{2\pi (N_0+S_0)B}\right]^{\frac{N}{2}}exp\left[ -\frac{1}{2(N_0+S_0)B}\sum_{k=1}^{N}z_k^2 \right]\\
	\end{aligned}
\]
因此似然比检测系统的判决准则为 \[
	\begin{aligned}
	\frac{P(z|H_1)}{P(z|H_0)}=\left(\frac{N_0}{N_0+S_0}\right)^{\frac{N}{2}}exp\left[\frac{S_0}{2(S_0+N_0)N_0B}\sum_{k=1}^{N}z_k^2\right]\mathop{\gtrless}\limits_{H_0}^{H_1} \lambda\\
	\end{aligned}
\]
即 \[
	l(z)=\sum_{i=1}^{2BT}z_k^2 \mathop{\gtrless}\limits ^{H_1}_{H_0} \left[\ln \lambda+BT\ln {(1+\frac{S_0}{N_0}})\right]\frac{2(S_0+N_0)N_0B}{S_0}=\gamma
\]
\item
考虑简单二元随机相位信号的检测问题,两个假设分别为\[
	\begin{aligned}
	&H_0:z(t)=n(t),& 0\le t\le T\\
	&H_1:z(t)=\sqrt{\frac{2E_s}{T}}\cos(\omega_0t+\theta)+n(t),&0\le t\le T\\
	\end{aligned}
\]
其中 $ E_s $是信号 $ s(t) =  \sqrt{\frac{2E_s}{T}}\cos(\omega_0t+\theta) $的能量; $ \omega_0 $是已知的常数; $ n(t) $是功率谱密度 $ G_n(\omega)=\frac{N_0}{2} $的高斯白噪声。假定相位 $ \theta $是概率密度函数为 \[
	p(\theta) = \frac{exp(v\cos \theta)}{2\pi I_0(v)},\quad -\pi\le \theta \le \pi
\]
的随机变量。若采用奈曼-皮尔逊准则,证明最佳检测系统如图题3-5所示。(注意,经过计算以及查找资料,发现书本上的图有错,下支路中线性滤波器应为 $ h(t)=\sqrt{\frac{T}{2}}\cos\omega_0(T-t) $,之后的增益系数应为$\frac{4E_s}{N_0^2}$)

\textbf{解:}\\
对于该系统, 假设\[\left\{ \begin{aligned}
  & {{y}_{1}}(t)=\int\limits_{0}^{T}{z(t)\cos {{\omega }_{0}}t}dt \\ 
 & {{y}_{2}}(t)=\int\limits_{0}^{T}{z(t)\sin {{\omega }_{0}}t}dt \\ 
\end{aligned} \right.\]
其似然比函数为\[\begin{aligned}
 \Lambda (x)
 &=\frac{p(z;\theta |{{H}_{1}})}{p(z|{{H}_{0}})} \\ 
 & =\frac{\int\limits_{-\pi }^{\pi }{p(z|{{H}_{1}};\theta )p(\theta )}d\theta }{p(z|{{H}_{0}})} \\ 
 & =\frac{\int\limits_{-\pi }^{\pi }{{{\left( \frac{1}{{{N}_{0}}\pi } \right)}^{\frac{N}{2}}}\exp \left\{ -\frac{1}{{{N}_{0}}}\int\limits_{0}^{T}{{{\left[ z(t)-s(t) \right]}^{2}}}dt \right\}p(\theta )}d\theta }{{{\left( \frac{1}{{{N}_{0}}\pi } \right)}^{\frac{N}{2}}}\exp \left\{ -\frac{1}{{{N}_{0}}}\int\limits_{0}^{T}{{{z}^{2}}(t)}dt \right\}} \\ 
 & =\int\limits_{-\pi }^{\pi }{\exp \left\{ -\frac{1}{{{N}_{0}}}\int\limits_{0}^{T}{\left[ -2z(t)s(t)+{{s}^{2}}(t) \right]}dt \right\}p(\theta )}d\theta  \\ 
 & =\int\limits_{-\pi }^{\pi }{\exp \left\{ -\frac{1}{{{N}_{0}}}\left[ \int\limits_{0}^{T}{-2z(t)s(t)}dt+{{E}_{s}} \right] \right\}p(\theta )}d\theta  \\ 
 & =\int\limits_{-\pi }^{\pi }{\exp \left\{ -\frac{1}{{{N}_{0}}}\left[ \int\limits_{0}^{T}{-2z(t)s(t)}dt+{{E}_{s}} \right] \right\}p(\theta )}d\theta  \\ 
 & =\exp \left( \frac{{{E}_{s}}}{{{N}_{0}}} \right)\int\limits_{-\pi }^{\pi }{\exp \left\{ \frac{2}{{{N}_{0}}}\sqrt{\frac{2{{E}_{s}}}{T}}\left[ {{y}_{1}}(t)\cos \theta -{{y}_{2}}(t)\sin \theta  \right] \right\}p(\theta )}d\theta  \\ 
 & =\frac{\exp ({{E}_{s}}/{{N}_{0}})}{2\pi {{I}_{0}}(v)}\int\limits_{-\pi }^{\pi }{\exp \left\{ M\left[ {{y}_{1}}(t)\cos \theta -{{y}_{2}}(t)\sin \theta  \right]+v\cos \theta  \right\}}d\theta  \\ 
 & =\frac{K}{2\pi {{I}_{0}}(v)}\int\limits_{-\pi }^{\pi }{\exp \left\{ M\left[ \left( {{y}_{1}}(t)+\frac{v}{M} \right)\cos \theta -{{y}_{2}}(t)\sin \theta  \right] \right\}}d\theta  \\ 
 & =\frac{K}{2\pi {{I}_{0}}(v)}\int\limits_{-\pi }^{\pi }{\exp \left[ A(t)\cos (\theta +\varphi ) \right]}d\theta  \\ 
 & =\frac{K}{{{I}_{0}}(v)}{{I}_{0}}(A(t)) \\ 
\end{aligned}\]
上式中, \[
	\begin{aligned}
	K = &exp(E_s/N_0)\\
	M=&\frac{2}{N_0}\sqrt{\frac{2E_s}{T}}\\
	A^2(t)=&{{\left[ M{{y}_{1}}(t)+v \right]}^{2}}+{{\left[ M{{y}_{2}}(t) \right]}^{2}}={{M}^{2}}\left[ y_{1}^{2}(t)+y_{2}^{2}(t) \right]+2Mvy_1(t)+v^2
	\end{aligned}
\]
根据贝塞尔函数 $ I_0(x) $的单调性,将 $ v^2 $减去,因此判决准则为 \[
\begin{aligned}
	{{M}^{2}}\left[ y_{1}^{2}(t)+y_{2}^{2}(t) \right]+2Mvy_1(t) \mathop{\gtrless} \limits ^{H_1}_{H_0}\lambda_0\\
\end{aligned}
\]
其中, $ \lambda_0 $根据奈曼-皮尔逊准则获得。\\
观察给的检测系统,上支路输出为 $ 2Mvy_1(t) $\\
下支先经过一个匹配滤波器,在经过一个包络平方检波器得到匹配滤波器包络的平方。\\
线性滤波器输出为 \[\begin{aligned}
 y(t)& =\sqrt{\frac{2}{T}}\int\limits_{0}^{t}{\cos {{\omega }_{0}}(T-t+\tau )z(\tau )}d\tau  \\ 
 & =\sqrt{\frac{2}{T}}\left[ \cos {{\omega }_{0}}(T-t)\int\limits_{0}^{t}{z(\tau )\cos {{\omega }_{0}}\tau }d\tau +\sin {{\omega }_{0}}(T-t)\int\limits_{0}^{t}{z(\tau )\sin {{\omega }_{0}}\tau }d\tau  \right] \\ 
\end{aligned}\]
经过平方检波器后输出为 \(\frac{2}{T}\left[ y_{1}^{2}(t)+y_{2}^{2}(t) \right]\),经过增益后输出\[{{M}^{2}}\left[ y_{1}^{2}(t)+y_{2}^{2}(t) \right]\]
该检测系统将上下支路相加之后即是之前所求的 $ A^2(t) -v^{2}$,将其和门限比较。符合之前的推倒,此系统为最佳检测系统。

证毕

%%%%%%%%%做题十分钟,打题半小时%%%%%%%%%%
%%%%%%%%%注意不要打开PDF后还编译%%%%%%%%%
\end{enumerate}
\end{document}